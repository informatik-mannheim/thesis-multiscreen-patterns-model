\newcommand{\hsmasprache}{de} % de oder en für Deutsch oder Englisch

% Titel der Arbeit auf Deutsch
\newcommand{\hsmatitelde}{Ein formales Modell zur Beschreibung geräteübergreifender Interaktionsmuster}

% Titel der Arbeit auf Englisch
\newcommand{\hsmatitelen}{A formal model for the description of interaction patterns between devices}

% Weitere Informationen zur Arbeit
\newcommand{\hsmaort}{Mannheim}    % Ort
\newcommand{\hsmaautorvname}{Horst} % Vorname(n)
\newcommand{\hsmaautornname}{Schneider} % Nachname(n)
\newcommand{\hsmadatum}{27.11.2015} % Datum der Abgabe
\newcommand{\hsmajahr}{2015} % Jahr der Abgabe
\newcommand{\hsmafirma}{Paukenschlag GmbH, Mannheim} % Firma bei der die Arbeit durchgeführt wurde
\newcommand{\hsmabetreuer}{Prof. Thomas Smits, Hochschule Mannheim} % Betreuer an der Hochschule
\newcommand{\hsmazweitkorrektor}{Prof. Kirstin Kohler, Hochschule Mannheim} % Betreuer im Unternehmen oder Zweitkorrektor
\newcommand{\hsmafakultaet}{I} % I für Informatik
\newcommand{\hsmastudiengang}{IM} % IB IMB UIB IM MTB
  
% Zustimmung zur Veröffentlichung
\setboolean{hsmapublizieren}{true}   % Einer Veröffentlichung wird zugestimmt
\setboolean{hsmasperrvermerk}{false} % Die Arbeit hat keinen Sperrvermerk

% -------------------------------------------------------
% Abstract

\newcommand{\hsmaabstractde}{Um Designern und Entwicklern zu ermöglichen, intuitiv nutzbare Übergänge zwischen verschiedenen Geräten in einem zusammenhängenden Nutzungskontext zu gestalten, wurde im Projekt SysPlace ein Katalog von Entwurfsmustern entwickelt. Bisher fehlt in diesen Entwurfsmustern allerdings eine einheitliche Möglichkeit, die technische Umsetzung der Interaktionen konsistent und formal zu beschreiben.

In der vorliegenden Arbeit wird mittels einer vergleichenden Literaturrecherche eine Abstraktion der technischen Aspekte erarbeitet, die geräteübergreifende Interaktionen implementieren. Das Ergebnis ist ein Modell, das eine formale Beschreibung der erarbeiteten Abstraktionen darstellt und in die Entwurfsmuster integriert werden kann. Anhand ausgewählter Entwurfsmuster wird die Verwendung des Modells veranschaulicht.}

\newcommand{\hsmaabstracten}{The SysPlace project developed a design pattern catalogue for intuitive transitions between devices, which allows designers and software developers to create a coherent user experience. The patterns lack a way to describe the interaction's technical implementation uniformly and consistently. 

In the scope of this thesis, we conducted a literature research in order to create an abstraction including the different technical aspects of interactions between devices. The result is a model, which allows a formal description of the suggested abstractions and can be integrated into existing design patterns. We illustrate the use of this model with selected design patterns.}