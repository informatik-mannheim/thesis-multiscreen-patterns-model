\documentclass[11pt,a4paper]{article}
\usepackage[utf8]{inputenc}
\usepackage[german]{babel}
\usepackage[T1]{fontenc}
\usepackage{amsmath}
\usepackage{amsfonts}
\usepackage{amssymb}
\author{}
\title{Exposé zur Abschlussarbeit}
\date{}
\hyphenation{In-ter-ak-ti-ons-mus-ter In-ter-ak-ti-on}
\hyphenation{In-ter-ak-ti-ons-kon-zep-te}
\hyphenation{ge-rä-te-ü-ber-grei-fend}
\hyphenation{Im-ple-men-tie-rungs-de-tails}
\hyphenation{In-ter-ak-ti-ons-ge-stal-tung}

\begin{document}
\maketitle\vspace{-1cm}

\begin{center}
\bgroup
\def\arraystretch{1.5}
\begin{tabular}{|l|p{8cm}|}
\hline 
Name & Schneider \\ 
\hline 
Vorname & Horst \\ 
\hline 
Matrikelnummer & 001324673 \\ 
\hline 
Studiengang & Informatik Master \\ 
\hline 
Erstgutachter & Prof. Smits \\ 
\hline 
Zweitgutachter & Prof. Kohler \\ 
\hline 
Titel (geplant) & Unterstützung der technischen Umsetzung von geräteübergreifenden Interaktionsmustern durch formale Modelle \\ 
\hline 
\end{tabular} 
\egroup
\end{center}

\hfill 
\break
\tolerance=450
Die meisten Menschen verbringen einen Großteil ihres Tages vor Bildschirmen verschiedenster Art, von Smartphones über Tablets zu Desktop-Rechnern und internetfähigen Fernsehern. Viele Aufgaben werden dabei auf unterschiedlichen Geräten gleichzeitig oder sequentiell ausgeführt, Anwendungen und Daten müssen zwischen diesen Geräten verteilt und synchronisiert werden. Verschiedene Interaktionskonzepte sind entstanden, um diese Geräteübergänge für den Anwender intuitiv nutzbar zu machen.\\
Aufgrund der heterogenen Darstellung und Umsetzung dieser Multiscreen-Konzepte entstand das Forschungsprojekt SysPlace, welches diese strukturiert und kategorisiert in Form von Interaktionsmustern erfasst und Designern sowie Entwicklern zur Verfügung stellt. 
Ziel dieser Arbeit soll es sein, die technischen Aspekte dieser verschiedenen Interaktionsmuster zu analysieren und eine formale Basis zu entwickeln, um diese einheitlich, abstrakt und plattformunabhängig beschreiben zu können. Dabei liegt der besondere Fokus auf der flexiblen Ausgestaltungsmöglichkeit von Interaktionsdetails.

\paragraph{}
Einzelne Interaktionskonzepte wurden hinsichtlich ihrer technischen Machbarkeit bereits intensiv wissenschaftlich untersucht und prototypisch umgesetzt, zumeist mit einer an eine bestimmte Technologie gekoppelten Lösung. Für die einfachere Umsetzung solcher Konzepte sind in letzter Zeit zudem verschiedene Frameworks veröffentlicht worden, die von der Technologie abstrahieren und die geräteübergreifende Interaktionsgestaltung selbst in den Vordergrund stellen sollen. Zusammen mit den bereits im Projekt SysPlace konzipierten Prototypen bilden diese Vorarbeiten eine breite Basis, um daraus allgemeine Regeln für die Implementierung von Interaktionen im Multiscreen-Kontext abzuleiten.

\paragraph{}
Das Ergebnis der Arbeit ist ein Modell, das eine abstrakte, einheitliche Beschreibung der technischen Realisierung verschiedener Interaktionsmuster ermöglicht. Es soll damit Entwicklern die Möglichkeit bieten, die erforderlichen Komponenten, Abläufe und Nutzerinteraktionen sowie deren semantischen Zusammenhänge zu verstehen und konkret umsetzen zu können. Um das zu erreichen, wird die Modellbeschreibung so gestaltet, dass sie sich einfach in ausführbaren Code übersetzen lässt. Ziel der Arbeit ist es dabei nicht, ein fertiges Framework zu entwickeln oder einzelne Interaktionsmuster mit konkreten Technologien zu realisieren. Zudem wird keine Bewertung vorgenommen, inwieweit sich einzelne Technologien zur Umsetzung bestimmter Aspekte von Interaktionsmustern eignen. \\
Zur Validierung des Modells wird ein repräsentativer Satz von Interaktionsmustern aus dem bisher identifizierten Katalog des SysPlace-Projektes genommen, das Modell auf diese angewendet und hinsichtlich der Vollständigkeit und Qualität der Interaktionsgestaltung bewertet. Zudem wird anhand einer prototypischen Implementierung die konkrete Umsetzbarkeit des Modells analysiert.

\paragraph{}
Das Anmeldedatum für die Arbeit ist der 01.06.2015. Für Literaturrecherche und einen ersten Gliederungsentwurf wird der erste Monat veranschlagt, die Ergebnisse werden anschließend mit dem Betreuer besprochen. Der Vergleich von Konzepten und das Herausarbeiten der Gemeinsamkeiten und eines Modellentwurfs wird in den folgenden 2 Monaten vorgenommen. Die Überprüfung des Modells und die prototypische Implementierung werden weitere 6 Wochen dauern. 
Neben der Besprechung der Gliederung ist geplant, dem Betreuer einen Monat vor Abgabe  einen Entwurf der Arbeit sowie frühzeitig ein exemplarisches Probekapitel vorzulegen. Zudem würde ich mir zwischendurch bei entscheidenden Arbeitsergebnissen ein kurzes Feedback des Betreuers einholen.

\end{document}